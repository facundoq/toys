\begin{myframe}
\frametitle{Reconocimiento de gestos (Tesis de Lic.)}

    \begin{columns}
    
    \begin{column}{0.3\textwidth}
    \centering
    \includegraphics[width=\textwidth]{img/esqueletos}
    \end{column}
    
    \begin{column}{0.02\textwidth}
    \centering
    =
    \end{column}
            
    \begin{column}{0.3\textwidth}
    \centering
    \includegraphics[width=\textwidth]{img/bag-of-arrows}
    \end{column}
    
    \begin{column}{0.02\textwidth}
    \centering
    $\rightarrow$
    \end{column}
    
    \begin{column}{0.3\textwidth}
    \centering
    \includegraphics[width=\textwidth]{img/3layers}
    \end{column}
  \end{columns}
   \end{myframe} 



\begin{myframe}{}
\centering
\resizebox{1\textwidth}{!}{
 \begin{tikzpicture}[rounded corners=2pt,very thin]
\shade[draw=pbblue,left color=green!70,right color=blue!70] (0pt, 0pt) rectangle ++ (13, 2);

\draw[color=normal text.fg!50]  
  (0pt, 0pt) rectangle (13,2) 
    node[pos=0.5,color=normal text.fg] {.};

\node[draw,text width=3cm,align=center,color=green!70]  (bio) at (2,-1) {{\Huge Bio}};
\node[draw,text width=3cm,align=center,color=blue!30] (comp) at (11,-1) {{\Huge Comp}};

    \draw[->,very thick] (bio) to[out=-15,in=-179] (comp);
   \draw[->,very thick] (comp) to[out=165,in=0] (bio);

    
\node[circle,fill,minimum size=5mm] (head) at (11,1.67) {};
\node[rounded corners=2pt,minimum height=1.3cm,minimum width=0.4cm,fill,below = 1pt of head] (body) at (11,1.43) {};
\draw[->,very thick]  (10.5,1.1) -- (9,1.1);


\draw[line width=1mm,round cap-round cap] ([shift={(2pt,-1pt)}]body.north east) --++(-90:6mm);
\draw[line width=1mm,round cap-round cap] ([shift={(-2pt,-1pt)}]body.north west)--++(-90:6mm);
\draw[thick,white,-round cap] (body.south) --++(90:5.5mm);

\node[draw,text width=3cm,align=center]  at (11,-3) {Redes artificiales};    
\node[draw,text width=3cm,align=center]  at (6.5,-3) {Modelos};  
\node[draw,text width=3cm,align=center]  at (2,-3) {Redes biológicas};    

  \end{tikzpicture}
}
\end{myframe}


\begin{myframe}
\centering
{\Huge ¡Gracias!}
\end{myframe}


\begin{myframe}
\frametitle{Codificación de reglas simbólicas (discretas) con Redes Neuronales Recurrentes}
\centering
\begin{columns}
    \begin{column}{0.47\textwidth}
    \centering
    \includegraphics[width=\textwidth]{img/dfa2}
    \end{column}
    
    \begin{column}{0.02\textwidth}
    \centering
    =
    \end{column}
    
    \begin{column}{0.47\textwidth}
    \centering
    \includegraphics[width=\textwidth]{img/recurrent2}
    \end{column}
\end{columns}
\end{myframe}




\begin{myframe}
\frametitle{Redes neuronales, profundas, convolucionales (2012)}
\centering


\includegraphics[width=1\textwidth]{img/convolutional}

\begin{block}{\centering Inspiradas por la corteza visual}
\centering
\begin{itemize}
\centering
\item Muchas capas
\item $\rightarrow$ dirección de mayor abstracción
\item Capas de convolución = filtros
\end{itemize}
\end{block}

\includegraphics[width=0.5\textwidth]{img/deepconvolutional}
\end{myframe}
