

\begin{myframe}
\pgfmathsetseed{1}
\centering
\resizebox{1\textwidth}{!}{
\begin{tikzpicture}[]
\begin{axis}[xlabel={Amplitud},ylabel={Duración}, yticklabels={,,},xticklabels={,,},]
\datosnoseparable
\node[color=surface] at (axis cs:1,1) { {\Huge ¿¿¿???} };
\end{axis}
\end{tikzpicture}
}
\end{myframe}


\begin{myframe}
\pgfmathsetseed{1}
\centering
\resizebox{1\textwidth}{!}{
\begin{tikzpicture}
   \begin{axis}[samples=20,yticklabels={,,},xticklabels={,,},zticklabels={,,},xlabel={Duración},ylabel={Amplitud},zlabel={Pendiente}]
   
    \addplot3[color=white, domain=-4:4] {-7*x-14*y+10};
     
     \addplot3[only marks, mark=o,mark options={negative},samples=10] ({invgauss(rnd,rnd)+2},{invgauss(rnd,rnd)+2},{invgauss(rnd,rnd)+2});
     
     \addplot3[only marks, mark=o,mark options={positive},samples=10] ({invgauss(rnd,rnd)-2},{invgauss(rnd,rnd)-2},{invgauss(rnd,rnd)-20});

   \end{axis}
\end{tikzpicture}
}
\end{myframe}

\begin{myframe}
\pgfmathsetseed{1}
\centering
\resizebox{1\textwidth}{!}{

\begin{tikzpicture}
   \begin{axis}[samples=20,yticklabels={,,},xticklabels={,,},zticklabels={,,},
   xlabel={Duración},ylabel={Amplitud},zlabel={Pendiente}]
     \addplot3[surf, domain=-4:4] {-7*x-14*y+10};

     \addplot3[only marks, mark=o,mark options={negative},samples=10] ({invgauss(rnd,rnd)+2},{invgauss(rnd,rnd)+2},{invgauss(rnd,rnd)+2});
     
     \addplot3[only marks, mark=o,mark options={positive},samples=10] ({invgauss(rnd,rnd)-2},{invgauss(rnd,rnd)-2},{invgauss(rnd,rnd)-20});

   \end{axis}
\end{tikzpicture}
}

\end{myframe}

\begin{myframe}    
\centering
%\documentclass{standalone}
%\usepackage{tikz}
%\usetikzlibrary{matrix,chains,positioning,decorations.pathreplacing,arrows}

%\begin{document}
    \tikzstyle{every node}=[font=\large]

\begin{tikzpicture}[
plain/.style={
  draw=none,
  fill=none,
  },
net/.style={
  matrix of nodes,
  nodes={
    draw,
    circle,
    inner sep=10pt
    },
  nodes in empty cells,
  column sep=2cm,
  row sep=4pt
  },
>=latex
]
\matrix[net] (mat)
{
$x$ & |[plain]| \\
$y$ & $ f$ \\
$z$ & |[plain]| \\
};

%\draw[<-] (mat-1-1) -- node[above left] {x} +(-2cm,0);
%\draw[<-] (mat-3-1) -- node[above left] {y} +(-2cm,0);


\draw[->] (mat-1-1) --  node[below] {$w_x$} (mat-2-2);
\draw[->] (mat-2-1) -- node[below] {$w_y$} (mat-2-2);
\draw[->] (mat-3-1) -- node[below] {$w_z$} (mat-2-2);
    
\draw[->] (mat-2-2) -- node[right] {$\; f=signo(w_x x + w_y y+ w_z z + b)$} +(1.2cm,0);

\end{tikzpicture}


%\end{document}
\end{myframe}



\begin{myframe}
\pgfmathsetseed{1}
\centering
\resizebox{1\textwidth}{!}{
\begin{tikzpicture}[]

  \node[draw,text width=3cm,align=center] (x1) at (0,0) {
    Espacio de la capa 1\\
    (de entrada)
  };
  \node[draw,text width=3cm,align=center] (x2) at (4,-2) {
      Espacio de la capa 1
      (oculta)
  };
  
  \node[draw,minimum height=1cm,text width=3cm,align=center] (x3) at (8,-4) {
        $\dots$
   };
  
    \node[draw,text width=3cm,align=center] (x4) at (12,-6) {
        Espacio de la capa N-1
        (oculta)
    };
     
\node[draw,text  width=3cm,align=center] (x5) at (16,-8) {
           Espacio de la capa N 
            (de salida)
    };
    
   \draw[->,very thick] (x1) to node[above right] {$T_1:: A \rightarrow B$} (x2);
    \draw[->,very thick] (x2) to node[above right] {$T_2
    :: B \rightarrow C$} (x3);
    \draw[->,very thick] (x3) to node[above right] {$T_{N-2} ::X \rightarrow Y $} (x4);
    \draw[->,very thick] (x4) to node[above right] {$T_{N-1}  :: Y  \rightarrow Z $} (x5);


\end{tikzpicture}
}
%\pause
\vspace{-4px}
\begin{block}{\centering Aprendizaje Automático = Estadística + Esteroides}
\centering
\begin{itemize}
\centering
\item Más \textbf{potencia} para encontrar asociaciones
\item PERO son modelos de \textbf{caja negra}
\end{itemize}
\end{block}
\end{myframe}
